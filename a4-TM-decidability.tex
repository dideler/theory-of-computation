% Copyright (C) 2011  Dennis Ideler (dennisideler.com)
%    Permission is granted to copy, distribute and/or modify this document
%    under the terms of the GNU Free Documentation License, Version 1.3
%    or any later version published by the Free Software Foundation;
%    with no Invariant Sections, no Front-Cover Texts, and no Back-Cover Texts.
%    A copy of the license is included in the section entitled "GNU
%    Free Documentation License" at http://www.gnu.org/copyleft/fdl.html

\documentclass[10pt,a4paper,final]{article}
\usepackage[latin1]{inputenc}
\usepackage{amsmath}
\usepackage{amsfonts}
\usepackage{amssymb}
\usepackage{graphicx}
\usepackage{upgreek} % for /Upphi
\setlength{\topmargin}{-.5in}
\setlength{\textheight}{9in}
\setlength{\oddsidemargin}{.125in}
\setlength{\textwidth}{6.25in}
\author{Dennis Ideler}
\title{MATH/COSC 4P61: Theory of Computation\\Assignment 4}
\begin{document}
\maketitle

\begin{enumerate}
\item % Q1
Construct a Turing Machine for $L = \{a^n b^n c^n | n \geq 1\}$. \\
\\
\begin{equation*}
M = (Q, \Sigma, \Upgamma, \delta, q_0, B, F)
\end{equation*}

\begin{tabular}{c l}
$Q:$ & finite set of states \\ 
$\Sigma:$ & input alphabet \\ 
$\Upgamma:$ & tape alphabet (contains $\Sigma$) \\
$\delta:$ & transition function \\
$q_0:$ & start state ($q_0 \in Q$) \\
$B:$ & blank symbol\footnotemark[1] ($B \in \Upgamma - \Sigma$) \\
$F:$ & set of final states ($F \subseteq Q$)
\end{tabular} 
\footnotetext[1]{All tape except for the input is blank initially.}

\begin{equation*}
\delta (q, Z) = (p, Y, D)
\end{equation*}

\begin{tabular}{c l}
$q:$ & state in $Q$ \\ 
$Z:$ & tape symbol in $\Upgamma$ \\ 
$p:$ & state in $Q$ \\
$Y:$ & new tape symbol in $\Upgamma$ \\
$D:$ & direction (L or R)
\end{tabular}

Each transition function, the state changes from $q$ to $p$, $Z$ is replaced by $Y$ on tape,
and the tape head moves one cell in the direction of $D$.

\underline{Idea:} Keep matching $a$'s, $b$'s, and $c$'s one by one, moving the head back and forth
along the tape. When all matchings are complete (in a legal string),
the head moves to a final state and halts. \\
\\
Let $w$ be a string of input symbols.
\begin{enumerate}
  \item If $w \in L(M)$, $M$ accepts $w$, $M$ halts in final state
  \item If $w \notin L(M)$, $M$ does not accept $w$,
  \begin{enumerate}
    \item $M$ halts on non-final state
    \item $M$ never halts (infinite loop)
  \end{enumerate}
\end{enumerate}

In pseudo-code, it would look something like this:
\begin{verbatim}
1. if you see an a, change it to X, and move R
   if you see a Y (no more a's), keep going R until B (final state)
2. keep going R until you see leftmost b, change it to Y, and move R
3. keep going R until you see leftmost c, change it to Z, and move L
4. keep going L until you see leftmost X, move R, go to step 1
\end{verbatim}

\begin{table}[h!] % add table because tabular environment is not a float by default (i.e. can be broken)
\begin{center}
\begin{tabular}{c| c c c c c c c}
$\delta$ & a        & b           & c           & X           & Y           & Z            & B \\
\hline
$q_0$ & $(q_1,X,R)$ & ---         & ---         & ---         & $(q_4,Y,R)$ & ---          & --- \\
$q_1$ & $(q_1,a,R)$ & $(q_2,Y,R)$ & ---         & ---         & $(q_1,Y,R)$ & ---          & --- \\
$q_2$ & ---         & $(q_2,b,R)$ & $(q_3,Z,L)$ & ---         & ---         & $(q_2,Z,R)$  & --- \\
$q_3$ & $(q_3,a,L)$ & $(q_3,b,L)$ & ---         & $(q_0,X,R)$ & $(q_3,Y,L)$ & $(q_3,Z,L)$  & --- \\
$q_4$ & ---         & ---         & ---         & ---         & $(q_4,Y,R)$ & $(q_4,Z,R)$  & $(q_f,B,R)$ \\
\hline
$q_f$ & ---         & ---         & ---         & ---         & ---         & ---          & ---
\end{tabular}
\end{center}
\caption{A Turing machine to accept $\{a^n b^n c^n | n \geq 1\}$}
\end{table}

$Q$ consists of
\begin{tabular}{c l}
$q_0:$ & change $a$'s to $X$ \\
$q_1:$ & change $b$'s to $Y$ \\
$q_2:$ & change $c$'s to $Z$ \\
$q_3:$ & backtrack the tape head to start a new round of matching \\
$q_4:$ & matching complete, go to end of input \\
$q_f:$ & final state, machine halted and input accepted
\end{tabular}

\begin{table}[h!]
\begin{center}
\begin{tabular}{|c|c|c|c|c|c|c|c|c|c|c|c|}
\hline
... & B & B & a & a & b & b & c & c & B & B & ... \\
\hline
... & B & B & X & a & b & b & c & c & B & B & ... \\
\hline
... & B & B & X & a & Y & b & c & c & B & B & ... \\
\hline
... & B & B & X & a & Y & b & Z & c & B & B & ... \\
\hline
... & B & B & X & X & Y & b & Z & c & B & B & ... \\
\hline
... & B & B & X & X & Y & Y & Z & c & B & B & ... \\
\hline
... & B & B & X & X & Y & Y & Z & Z & B & B & ... \\
\hline
\end{tabular}
\end{center}
\caption{An example of how the tape would change if $w = aabbcc$ (not showing states and skipping steps
where a symbol is replaced by the same symbol)}
\end{table}

\item % Q2
Let $ALL_{CGF} = \{<G> |\, G$ is a CFG and $L(G) = \Sigma^*\}$.
It is known that this language is undecidable.
Define $EQ_{CFG} = \{<G,H> |\, G$ and $H$ are CFG's and $L(G) = L(H)\}$.
Show that $EQ_{CFG}$ is undecidable. \\
\\
We have to use reduction. Show we can solve $ALL_{CFG}$, then we can solve $EQ_{CFG}$.
But $ALL_{CFG}$ is known to be unsolvable, so then $EQ_{CFG}$ is also unsolvable since
$ALL_{CFG} \leq EQ_{CFG}$. $ALL_{CFG}$ reduces to $EQ_{CFG}$, so $EQ_{CFG}$ is at least as hard
as $ALL_{CFG}$. \\
\\
\emph{TODO}

\item % Q3
Given $A = \{<M> |\, M$ is a finite automaton and $L(M) = \Phi\}$
where $M$ is some encoding of the machine $M$. Is $A$ Turing-decidable? Prove your answer. \\
\\
\emph{TODO}

\item % Q4
If $L$ is a recursively enumerable language, what can you say about its complement?
Is it recursive, recursively enumerable, or non-recursively enumerable? Prove your answer. \\
\\
There are two cases:
\begin{enumerate}
  \item $L$ is recursive
  \item $L$ is not recursive
\end{enumerate}
If $L$ is recursive, then so is $\bar{L}$, because recursive languages are closed under complementation.
That means it is Turing-decidable, thus there is a TM that can solve it
(i.e. it will stop either by accepting or rejecting). \\
\\
If $L$ is not recursive (but it is recursively-enumerable)...\\
Assume that recursively enumerable languages are closed under complement.
Then $\bar{L}$ is recursively enumerable, which means it is Turing-recognizable.
Thus there is a TM that will stop eventually on a final state if fed a legal string. 

\begin{verbatim}
                                 _ _
L(M)    /--y--> Y                L(M)     /--y--> N
    +-+/                              +=+/
w-->|M|                           w-->|M|
    +-+                               +-+
\end{verbatim}
W see, that's not the case because $M$ only needs to halt if $w \in L(M)$, it doesn't have
to say ``No".
In other words, assuming recursively enumerable languages are closed under complementation,
then if $w \in L$, the TM that recognizes $\bar{L}$ should also accept $w$. But that TM
only accepts $w \notin L$, so we have a contradiction.

For reference, recursively enumerable languages are closed under the following properties:
\begin{enumerate}
  \item Kleene closure ($L^*$ of $L$)
  \item concatenation ($L \cdot P$)
  \item union ($L \cup P$)
  \item intersection ($L \cap P$)
\end{enumerate}
Recursively languages are \underline{not} closed under:
\begin{enumerate}
  \item set difference ($L - P$)
  \item complementation ($\bar{L}$ of $L$)
\end{enumerate}
Which means the resulting languages produced by those 2 properties are not \emph{guaranteed} to be
recursively enumerable.

\item % Q5
Let $M$ be a deterministic Turing maching that accepts a non-recursive language.
Prove that the halting problem for $M$ is undecidable.
That is, there is no TM that takes input $w$ and determines whether the computation of $M$ halts with
with input $w$. \\
\\
If the halting problem for $M$ was decidable, then there would be a general TM (or algorithm)
that given a description of a computer computer (or better yet, a non-recursive language among a set
of uncountable ones) that decides whether the program finishes running or continues to run forever.

Alan Turing proved in 1936 that such a general algorithm to solve the halting problem for \emph{all}
possible program-input pairs cannot exist.

We know that the number of TM's is countable, but the number of languages is uncountable
(because there is an uncountable number of non-recursively enumerable languages). So
some problems are unsolvable (i.e. not decidable).

We can use the technique of diagonalization. Assume that our TM can decide on the halting problem
for any arbitrary $w$ and $M$. So we can enumerate for every program-input pair the decision (yes or no). But we can show that there are programs that are not in our enumeration (and it makes sense because
$M$ is from an uncountably number of non-recursively enumerable languages).

So we have a program that doesn't appear in our enumeration but it should. Our contradiction means
that the halting problem for $M$ is undecidable.
\end{enumerate}

\end{document}
